\documentclass{article}

\usepackage[T1]{fontenc}
\usepackage[osf]{libertine}
\usepackage[scaled=0.8]{beramono}
\usepackage[margin=1.5in]{geometry}
\usepackage{url}
\usepackage{amsmath}
\usepackage{amssymb}
\usepackage{nicefrac}
\usepackage{microtype}
\usepackage{subcaption}
\usepackage{bm}

\usepackage{sectsty}
\sectionfont{\large}
\subsectionfont{\normalsize}

\usepackage{titlesec}
\titlespacing{\section}{0pt}{10pt plus 2pt minus 2pt}{0pt plus 2pt minus 0pt}
\titlespacing{\subsection}{0pt}{5pt plus 2pt minus 2pt}{0pt plus 2pt minus 0pt}

\usepackage{pgfplots}
\pgfplotsset{
  compat=newest,
  plot coordinates/math parser=false,
  tick label style={font=\footnotesize, /pgf/number format/fixed},
  label style={font=\small},
  legend style={font=\small},
  every axis/.append style={
    tick align=outside,
    clip mode=individual,
    scaled ticks=false,
    thick,
    tick style={semithick, black}
  }
}

\pgfkeys{/pgf/number format/.cd, set thousands separator={\,}}

\usepgfplotslibrary{external}
\tikzexternalize[prefix=tikz/]

\newlength\figurewidth
\newlength\figureheight

\setlength{\figurewidth}{12cm}
\setlength{\figureheight}{6cm}

\newlength\squarefigurewidth
\newlength\squarefigureheight

\setlength{\squarefigurewidth}{4cm}
\setlength{\squarefigureheight}{4cm}

\newlength\smallsquarefigurewidth
\newlength\smallsquarefigureheight

\setlength{\smallsquarefigurewidth}{3.25cm}
\setlength{\smallsquarefigureheight}{3.25cm}

\newlength\smallfigurewidth
\newlength\smallfigureheight

\setlength{\smallfigurewidth}{6.25cm}
\setlength{\smallfigureheight}{4cm}

\setlength{\parindent}{0pt}
\setlength{\parskip}{1ex}

\newcommand{\acro}[1]{\textsc{\MakeLowercase{#1}}}
\newcommand{\given}{\mid}
\newcommand{\mc}[1]{\mathcal{#1}}
\newcommand{\data}{\mc{D}}
\newcommand{\intd}[1]{\,\mathrm{d}{#1}}
\newcommand{\inv}{^{-1}}
\newcommand{\trans}{^\top}
\newcommand{\mat}[1]{\bm{\mathrm{#1}}}
\renewcommand{\vec}[1]{\bm{\mathrm{#1}}}
\newcommand{\R}{\mathbb{R}}
\renewcommand{\epsilon}{\varepsilon}

\DeclareMathOperator{\var}{var}
\DeclareMathOperator{\cov}{cov}
\DeclareMathOperator{\diag}{diag}
\DeclareMathOperator*{\argmin}{arg\,min}
\DeclareMathOperator*{\argmax}{arg\,max}

\begin{document}

\section*{The Kernel Trick}

Consider the assumption of linear regression with an explicit feature
expansion $\phi\colon \vec{x} \mapsto \phi(\vec{x})$:
\begin{equation*}
  y(\vec{x}) = \phi(\vec{x})\trans \vec{w} + \epsilon(\vec{x}).
\end{equation*}
Given training data $\data = \bigl\{ (\vec{x}, y) \bigr\} = (\mat{X},
\vec{y})$, recall we modeled the residuals $\epsilon(\vec{x})$ as
zero-mean independent, identically distributed Gaussians with variance
$\sigma^2$:
\begin{equation*}
  p(\vec{\epsilon})
  =
  \mc{N}(\vec{\epsilon}; \vec{0}, \sigma^2 \mat{I}),
\end{equation*}
giving rise to the following likelihood:
\begin{equation*}
  p(\vec{y} \given \mat{X}, \vec{w}, \sigma^2)
  =
  \mc{N}(\vec{y};\mat{\Phi}\trans \vec{w}, \sigma^2 \mat{I}),
\end{equation*}
where we have defined $\mat{\Phi} = \phi(\mat{X})$.

In Bayesian linear regression, we further chose a multivariate
Gaussian prior for $\vec{w}$:
\begin{equation*}
  p(\vec{w}) = \mc{N}(\vec{w}; \vec{\mu}, \mat{\Sigma}).
\end{equation*}
For simplicity, in the below we assume the prior mean for $\vec{w}$ is
$\vec{\mu} = \vec{0}$.

Given these assumptions, we were able to derive the posterior
distribution of $\vec{w}$ given the data $\data$:
\begin{equation*}
  p(\vec{w} \given \data, \sigma^2)
  =
  \mc{N}(\vec{w};
  \vec{\mu}_{\vec{w}\given\data},
  \mat{\Sigma}_{\vec{w}\given\data}
  ),
\end{equation*}
where
\begin{align*}
  \vec{\mu}_{\vec{w}\given\data}
  &=
  \mat{\Sigma}
  \mat{\Phi}\trans
  (\mat{\Phi}\mat{\Sigma}\mat{\Phi}\trans + \sigma^2 \mat{I})\inv
  \vec{y};
  \\
  \mat{\Sigma}_{\vec{w}\given\data}
  &=
  \mat{\Sigma}
  -
  \mat{\Sigma}
  \mat{\Phi}\trans
  (\mat{\Phi}\mat{\Sigma}\mat{\Phi}\trans + \sigma^2 \mat{I})\inv
  \mat{\Phi}
  \mat{\Sigma}.
\end{align*}

If we wish to use our model to predict the outputs $\vec{y}_\ast$
associated with a set of inputs $\mat{X}_\ast$, we previously derived:
\begin{equation*}
  p(\vec{y}_\ast \given \mat{X}_\ast, \data, \sigma^2)
  =
  \mc{N}(
  \vec{y}_\ast;
  \mat{\Phi}_\ast \vec{\mu}_{\vec{w}\given\data},
  \mat{\Phi}_\ast \mat{\Sigma}_{\vec{w}\given\data} \mat{\Phi}_\ast\trans + \sigma^2 \mat{I}).
\end{equation*}
Examining the forms of these expressions, we see that the feature
expansion $\phi$ always appears in one of the following expressions:
\begin{equation*}
  \mat{\Phi}\mat{\Sigma}\mat{\Phi}\trans
  \qquad
  \mat{\Phi}_\ast\mat{\Sigma}\mat{\Phi}\trans
  \qquad
  \mat{\Phi}\mat{\Sigma}\mat{\Phi}_\ast\trans
  \qquad
  \mat{\Phi}_\ast\mat{\Sigma}\mat{\Phi}_\ast\trans.
\end{equation*}
The entries of these matrices are always of the form
$\phi(\vec{x})\trans \mat{\Sigma} \phi(\vec{x}')$, where $\vec{x}$ and
$\vec{x}'$ are two arbitrary inputs.  To simply our expressions, we
define a function
\begin{equation*}
  K(\vec{x}, \vec{x}')
  =
  \phi(\vec{x})\trans\mat{\Sigma}\phi(\vec{x}').
\end{equation*}
Because $\mat{\Sigma}$ is positive definite, it has a ``matrix square
root,'' $\mat{\Sigma}^{\nicefrac{1}{2}}$ with the property
$(\mat{\Sigma}^{\nicefrac{1}{2}})^2 = \mat{\Sigma}$.\footnote{You can
  prove this via the singular value decomposition (\acro{SVD}): write
  $\mat{\Sigma} = \mat{U}\mat{D}\mat{U}\trans$, where $\mat{U}$ is
  unitary and $\mat{D}$ is diagonal with positive entries (because
  $\mat{\Sigma}$ is positive definite), then define
  $\mat{\Sigma}^{\nicefrac{1}{2}} =
  \mat{U}\mat{D}^{\nicefrac{1}{2}}\mat{U}\trans$.} If we define the
function $\psi(\vec{x}) =
\mat{\Sigma}^{\nicefrac{1}{2}}\phi(\vec{x})$, we can see that
$K$ is simply an inner product:
\begin{equation*}
  K(\vec{x}, \vec{x}')
  =
  \psi(\vec{x})\trans \psi(\vec{x}').
\end{equation*}
Such a function $K$ is guaranteed to always produce positive-definite
Gram matrices (a \emph{Gram matrix} is a square matrix of inner
products between pairs of elements), and is called a \emph{kernel} or
\emph{covariance function.}

Sometimes it is possible to specify a covariance function $K$ directly
without ever computing the feature map explicitly.  With such a
function, we could perform efficient Bayesian linear regression even
with a high-dimensional (or even infinite dimensional!) feature
expansion $\phi$ \emph{implicitly.} This idea of computing inner
products in a feature space directly is called the \emph{kernel trick}
and has been the basis of a large amount of work in the
machine-learning community.  Effectively, any algorithm that operates
purely in terms of inner products between input vectors can be made
nonlinear by replacing normal inner products with the evaluation of a
kernel.

With this definition, we may rewrite the predictive distribution
for $\vec{y}_\ast$:
\begin{equation*}
  p(\vec{y}_\ast \given \mat{X}_\ast, \data, \sigma^2)
  =
  \mc{N}(
  \vec{y}_\ast;
  \mu_{\vec{y}_\ast \given \data},
  K_{\vec{y}_\ast \given \data}),
\end{equation*}
where
\begin{align*}
  \vec{\mu}_{\vec{y}_\ast\given\data}
  &=
  K(\mat{X}_\ast, \mat{X})
  \bigl(K(\mat{X}, \mat{X}) + \sigma^2 \mat{I}\bigr)\inv
  \vec{y};
  \\
  K_{\vec{y}_\ast\given\data}
  &=
  K(\vec{X}_\ast, \vec{X}_\ast)
  -
  K(\mat{X}_\ast, \mat{X})
  \bigl(K(\mat{X}, \mat{X}) + \sigma^2 \mat{I}\bigr)\inv
  K(\mat{X}, \mat{X}_\ast).
\end{align*}

\subsection*{Examples}

Perhaps the most-commonly used kernel is the \emph{squared exponential
  covariance function:}
\begin{equation*}
  K(\vec{x}, \vec{x}'; \lambda, \ell)
  =
  \lambda^2
  \exp\biggl(-\frac{\lVert \vec{x} - \vec{x}' \rVert^2}{2\ell^2}\biggr),
\end{equation*}
where $\lambda$ and $\ell$ are parameters.  The former is simply a
multiplicative scaling constant (you can think of this as an implicit
scalar multiplication in the implicit feature map $\phi$).  The latter
takes the role of a \emph{length scale;} vectors separated by more
than a couple length scales will have a kernel value near zero.

An example of Bayesian linear regression using this kernel function is
shown in Figure \ref{kernel_example}.  We see that the use of this
kernel function allowed us to achieve nice nonlinear regression
without computing explicit basis expansions.  In fact, you can show
that the squared exponential kernel corresponds to an
\emph{infinite-dimensional} basis expansion, where we use a Gaussian
basis function \emph{centered on every point.}  Such a feature
expansion would be impossible to use if we attempted to use explicit
feature computation.

\begin{figure}
  \centering
  \input{figures/kernel_example.tex}
  \caption{Example of Bayesian linear regression using the squared
    exponential covariance function.  The true function is $f =
    \sin(x)$.  The kernel parameters are $\lambda = \ell = 1$, and the
    noise variance was set to $\sigma^2 = 0.1^2$.}
  \label{kernel_example}
\end{figure}

Sometimes thinking in terms of the kernel can help even when you have
an explicit feature expansion on hand.  As an example, imagine our
inputs are binary vectors of length $n$ (so each input $\vec{x}$ is a
subset, a member of the power set $\mc{P}(n)$).  One rather expensive
feature expansion we could try would be to enumerate every member of
$\mc{P}(n)$ and define $\phi(\vec{x})_i = \vec{s}_i \subset \vec{x}$,
where $\vec{s}_i$ is the $i$th element of the power set.  So we
represent our set $\vec{x}$ by a feature vector of length $2^n$
indicating every subset of $\vec{x}$.  This is a very expensive
feature expansion, requiring exponential space to store for each
input.  However, if we take $\mat{\Sigma} = \mat{I}$, we can compute
the dot product as:
\begin{equation}
  K(\vec{x}, \vec{x}') = 2^{\lvert x \cap x' \rvert},
\end{equation}
which only requires time and space linear in $n$!

\end{document}
